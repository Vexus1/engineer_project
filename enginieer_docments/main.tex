\documentclass[a4paper, 10pt]{article}
\usepackage{polski}
\usepackage[utf8]{inputenc}
\usepackage[T1]{fontenc}
\usepackage{geometry}
\usepackage{indentfirst}
\usepackage{titlesec}

\geometry{top=2cm, bottom=2cm, left=2.5cm, right=2.5cm}

% Dostosowanie odstępów dla sekcji i podsekcji
\titleformat{\section}[block]{\normalfont\Large\bfseries}{\thesection}{1em}{}
\titlespacing*{\section}{0pt}{3em}{2em} % Odstęp przed i po sekcji
\titleformat{\subsection}[block]{\normalfont\large\bfseries}{\thesubsection}{1em}{}
\titlespacing*{\subsection}{0pt}{2em}{1em} % Odstęp przed i po podsekcji

\begin{document}

    \thispagestyle{empty}

    \begin{center}
        \textbf{\LARGE Politechnika Wrocławska} \\[1em]
        \Large Wydział Matematyki \\[2em]

        \textbf{KIERUNEK:} \\
        \Large Matematyka Stosowana \\[2em]

        \textbf{\LARGE PRACA DYPLOMOWA \\[0.5em] INŻYNIERSKA} \\[6em]

        \textbf{\large TYTUŁ PRACY:} \\[1em]
        \textbf{\Large Analiza efektywności metod uczenia przez wzmacnianie w grach komputerowych} \\[2em]

        \textbf{\large AUTOR:} \\[1em]
        \textbf{\Large Adrian Galik} \\[2em]

        \textbf{\large PROMOTOR:} \\[1em]
        \textbf{\Large dr hab. Janusz Szwabiński } \\[10em]

        WROCŁAW 2024
    \end{center}

    \newpage
    \section{Wstęp}
    \indent Rozwój technologii w tempie przekraczającym wszelkie oczekiwania oraz zwiększająca się dostępnosć mocy
    obliczeniowej doprowadziły do tego że algorytmy uczenia maszynowego stanowią nieoderwalną część życia codziennego
    każdego z nas. Zastosowanie ich można znaleść w dziedzinach robotyki, rozpoznawania obrazów, przetwarzania
    języka naturalnego, klasyfikacja spamu, systemy nawigacyjne, diagnostyka chorób, sztuczna inteligencja w grach oraz
    wiele innych gałęzi technologii które oddziałują na nas w sposób pośredni lub bezpośredni. Jedną z najbardziej 
    fascynujących, a zarazem najstarszych dziedzin uczenia maszynowego jest uczenie przez wzmacnianie. Znana już od lat 50 ubiegłego wieku
    będzie ona kluczowym działem z którego algorymy będą stanowiły fundament mojej pracy.
    \newline
    \indent Celem niniejszej pracy inżynierskiej jest analiza efektywności wybranych metod uczenia
    przez wzmacnianie w grach komputerowych. Przede wszystkim badania oraz porównania algorytmów zarówno jeśli chodzi o czas uczenia
    oraz efektywność zostały przeprowacone na przykładzie gry Pong, która jest bardzo często wykorzystywana jako dobry przykład środowiska testowego
    do badań nad algorytmami sztucznej inteligencji. W ramach pracy zaimplementowałem trzy popularne metody uczenia przez wzmacnianie:
    Deep Q-Learning (DQN), Advantage Actor-Critic (A2C) oraz Asynchronous Advantage Actor-Critic (A3C), a w następnym kroku zbadałem ich efektywność
    na zasadzie różnych parametrów m. in. prędkość uczenia oraz skuteczność gry.

    \section{Wprowadzenie do uczenia maszynowego}
    \indent Uczenie maszynowe jest jedną z kluczowych gałęzi sztucznej inteligencji, której celem jest tworzenie algorytmów zdolnych
    do uczenia się na podstawie danych i podejmowania decyzji bez konieczności programowania reguł działania. 
    Oto nieco ogólniejsza definicja: Uczenie maszynowe to "dziedzina nauki dająca komputerom możliwość uczenia się
    bez konieczności ich jawnego programowania". - Arthur Samuel, 1959. A tu bardziej techniczna:
    "Mówimy, że program komputerowy uczy się na podstawie doświadczenia E w odniesieniu do jakiegoś zadania T
    i pewnej miary wydajności P, jeśli jego wydajność (mierzona przez P) wobec zadania T wzrasta wraz z nabywaniem
    doświadczenia E". - Tom Mitchell, 1997. Przykładowe dane używane do trenowania systemu noszą nazwę
    \textbf{zbioru/zestawu uczącego} (ang. training set). Każdy taki element uczący jest nazywany 
    \textbf{przykładem uczącym (próbką uczącą)}. Część systemu uczenia maszynowego odpowiedzialna za uczenie się i uzyskiwanie 
    przwidywań nazywana jest modelem. Przykładowymi modelami są sieci neuronowe i lasy losowe. Dla przykładu klasyfikacji spamu to zgodnie z definicją Toma Mitchella: naszym
    zadaniem T jest oznaczenie spamu, doświadczeniem E - dane uczące a do wyznaczenia pozostaje miara wydajności P.
    Może być nią na przykład stosunek prawidłowo oznaczonych wiadomości do przykjładów nieprawidłowo zaklasyfikowanych.
    (książka uczenie maszynowe z użyciem Scikit-Learn, Keras i TensorFlow (5 zdań ostatnich))
    
    \subsection{Podział uczenia maszynowego} (Można dodać do każdego jakieś wykresy)
    Algorytmy uczenia maszynowego można podzielić na cztery ogólne kategorie:
    
    \subsubsection{Uczenie nadzorowane}
    To najczęstszy przypadek uczenia maszynowego. W tym przypadku algorytm uczy się na podstawie
    oznaczonych danych wejściowych które są opisane przez człowieka oraz odpowiadających im wyników.
    Głównymi zastosowaniami algorytmów uczenia nadzorowanego to klasyfikicja i regresja. Klasycznym przykładem
    jest klasyfikacja spamu, polega ona na analizie przez algorytm e-maila i przypisanie do niego kategorii "spam" 
    lub "nie spam". Przykład algorytmów: regresja liniowa, drzewa decyzyjne, SVM

    \subsubsection{Uczenie nienadzorowane}
    Algorytm analizuje dane bez użycia jakichkolwiek oznacznień w celu znalezienie grup lub ukrytych wzorców.
    Kluczowymi zadaniami uczenia nienadzorowanego są między innymi: wizualizacja danych,
    redukcja wymiarowości, analiza skupień, wyrywanie anomalii, wykrywanie nowości, usuwanie szumu
    oraz uczenie przy użyciu reguł asocjacyjnych. Przykład algorytmów: K-Means DBSCAN

    \subsubsection{Uczenie częściowo nadzorowane}
    Jest to specyficzny przypadek uczenia nadzorowanego, lecz ma ono na tyle odmienne zasady działania że tworzy
    odzielną kategorię. W uczeniu częściowo nadzorowanym algorytm nie używa oznaczeń nadanych przez człowieka, lecz są one 
    wygenerowane na podstawie danych wejściowych (zazwyczaj stosowane są do tego algorytmy heurystyczne).
    Jest to szczególnie przydatne w sytuacjach, gdy oznaczanie danych jest kosztowne lub czasochłonne jak przykładowo
    w diagnos~tyce medycznej. 

    \subsubsection{Uczenie przez wzmacnianie}
    Dziedzina która była zaniedbywana do momentu w którym autorzy projektu Google DeepMind wykorzystali ją w celu nauki komputerów 
    gier Atari. Jest to specyficzna forma uczenia maszynowego gdyż w zasadniczy sposób różni się od wszystkich poprzednich metod gdyż
    alogrytm nie uczy się za pomocą danych lecz na podstawię interakcji z dynamicznym środowiskiem stąd nazwa "wzmacnianie". 
    Agentem nazywamy element który jest odpowiedzialny za interakcję ze środowiskiem, a same interakcje nazywamy akcjami.
    Algorytm za wykonanie każdej akcji definiowanej przez autora otrzymuje adekwatnie do oczekiwań nagrodę i karę. Na podstawię 
    tej metody algorytm uczy się strategii która pozwala mu maksymalizować nagrodę na podstawię konkretnego stanu środowiska.
    
    \section{Teoretyczne podstawy uczenia przez wzmacnianie}

    \subsection{Podstawowe pojęcia i definicje}
    \begin{itemize}
        \item Agent - 
    \end{itemize}

\end{document}
